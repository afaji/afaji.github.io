\section*{Service Statement}

\subsection*{Service for Institution}

have prioritized service to my institution and the NLP department through various roles, contributing significantly to the academic and research environment at MBZUAI. Since joining, I have examined 5 PhD candidacy exams (for 2nd-year students) and 9 master's thesis defenses, serving not only my own NLP department but also students from the ML department. In particular, when the NLP department produced its first PhD graduate in 2025, I was honored to serve as one of the examiners.

I served as a member of the 1st-year PhD Qualification Exam committee in 2023 and 2025, tasked with writing and grading exams. Similarly, I served as a member of the MBZUAI online admission exams committee in 2023 and 2025. In this committee, we evaluate and renew the online examinations that MBZUAI uses for the admission process by providing new sets of questions and evaluating the old ones. I find exam committee work intellectually rewarding, as it consistently refreshes my own foundational understanding.

In a more extensive capacity, I served on various admission committees. Specifically, I served as a member for the 2024-2025 and chair for the 2025-2026 NLP admission committee, and as the chair for the 2025-2026 undergraduate admission committee. In this committee, we evaluate the' CVs of the applicants and conduct interviews. As chair, I have revised and streamlined the process to ensure consistency and scalability; this involved designing clear guidelines for committee decision making and updating interview protocols.

Additionally, I served as a faculty advisor for the \href{https://mbzuai.ac.ae/training_program/executive-program/}{MBZUAI Executive Education Program} (MEP) Cohort 4 in summer 2023. The MEP is a 5-month training initiative for leaders in industry and the public sector, culminating in an AI capstone project. My role involved guiding participants through their learning journey and providing feedback on their proposals. Beyond these roles, I have contributed through institutional talks, including student orientation for internships and sessions on industry vs. academia careers.

\subsection*{Service for Scientific Community}

Beyond institutional service, I contribute to the broader academic and research communities. I have served as a reviewer and area chair for top-tier conferences such as ACL, EMNLP, NAACL, ICLR, NeurIPS, and ACL Rolling Review, as well as several workshops. In 2025, I served as a member of COLING 2025 local chairs, whose job is to find sponsor, venue, local attraction, and other relevant stuff in Abu Dhabi.

I have also actively engaged with the NLP community, mostly relevant to my research direction and my research communities, by giving talks and keynotes at major conferences, including ACL, NAACL, EMNLP and COLING. My presentations primarily focus on inclusive data collection and multilingualism, with keynotes delivered at workshops such as WiNLP, CLTW, CALCS, and Field Matters. Furthermore, I have contributed to community building and technical education through the SEACrowd Birds of a Feather session at ACL 2024 and by teaching efficient model training techniques at the Mexican NLP Summer School during NAACL 2024.

I have also organized several workshops to support research in underrepresented areas. I organized \href{https://aclanthology.org/2023.sealp-1.0/}{SEALP 2023} and \href{https://sealp-workshop.github.io/program/}{SEALP 2025}, a workshop series dedicated to Southeast Asian languages, and I organized \href{https://melt-workshop.github.io/cfp/}{MELT 2025}, a new workshop focused on multilingual NLP, co-located in COLM. Currently, I am organizing the \href{https://multilingual-multicultural-evaluation.github.io/}{Multilingual Multicultural Evaluation} workshop for EACL 2026.

Beyond workshops, I also organized several SemEval shared tasks, as I find the creation of robust datasets and benchmarks to be a particularly rewarding aspect of research. In 2024, I helped organize \href{https://github.com/semantic-textual-relatedness/Semantic_Relatedness_SemEval2024}{Task 1}, which focused on measuring semantic relatedness for African and Asian languages, and \href{https://github.com/mbzuai-nlp/SemEval2024-task8}{Task 8}, which challenged participants to detect machine-generated text across different domains. In 2025, I co-organized \href{https://github.com/emotion-analysis-project/SemEval2025-Task11}{Task 11}, focusing on emotion detection for low-resource languages. These community challenges were well-received, \textbf{attracting around 160, 280, and 800 participants}, respectively.


\subsection*{Service for Grassroots NLP Community}

Indonesia has the 4th largest population in the world, yet it has historically been underrepresented in global research, including NLP. When I noticed a surge of Indonesian NLP activity around 2020, I was highly encouraged. I began collaborating with local researchers, which led to several joint projects and papers. As our network grew, I started connecting with researchers across the broader Southeast Asian region who faced similar challenges. We began collaborating on regional resources and papers, further widening our circle.

Our largest collaborative work thus far was SEACrowd, a dataset catalog for the region, which was inspired by NusaCrowd, an Indonesian variant on which we previously worked with the Indonesian NLP community. As we grew even bigger, we established the \href{https://seacrowd.org/}{SEACrowd} (Yes, the same name as the project's name), a non-profit organization to host researchers interested in the region and foster collaboration on meaningful projects. I currently serve on the Advisory Board and remain actively involved. We organize Birds-of-a-Feather sessions at major conferences and host our own workshop, \href{https://sealp-workshop.github.io}{SEALP}. We also work on collaborative initiatives; currently, we are building a Southeast Asian Vision-Language Model. Not only do we work with my local communities, we collaborate widely with other communities outside of South East Asia.

Since 2024, SEACrowd has also run an \href{https://seacrowd.org/apprenticeship.html}{apprenticeship program} where junior researchers work on projects mentored by senior members, with the aim of training the next generation of scientists. In the pilot batch, we had 3 projects where I was also a mentor. We learned valuable lessons from this pilot. Specifically, the need to manage project scope, as overambitious goals caused some students to lose momentum, and the importance of reducing team size to keep it focused. We are now opening the second batch, the first to involve public registration and I am happy to serve as a mentor again.

Personally, I find SEACrowd to be a vital community. It also has effectively become a talent pipeline for my group, with several Research Assistants and students joining MBZUAI through this network. Even beyond recruitment, it provides extensive networking opportunities, and I frequently involve my lab in SEACrowd collaborative projects.

\subsection*{Service for Olympiads and Competitive Programming Community}

Competitive programming has played a huge role in my career. I started practicing in 2007, which led to a Silver Medal at the International Olympiad in Informatics (IOI) in 2010. This achievement gave me a head start in computer science and the privilege of scholarships for my studies abroad. Grateful for these opportunities, I have dedicated many years to giving back to the community.

I started coaching early, leading clubs during my high school and undergraduate years. In 2008, I started teaching my high school classmates for competition preparation. I have since trained several competitive programming teams, including Indonesia’s national team (2011–2013), the University of Edinburgh’s ICPC team (2014), and Saudi Arabia’s national team for IOI (2020). In 2017, I co-wrote a \href{https://toki.id/buku-pemrograman-kompetitif-dasar/}{book} and created \href{https://tlx.toki.id/courses}{online training materials} that are now used as the official preparatory course for Indonesia’s National Olympiad. Currently, I am training the first batch of MBZUAI undergraduates for the ICPC. We won a silver medal at the Africa and Arab Regional Championship. This achievement qualifies us for the World Finals in 2026.

Designing interesting and challenging problems is not trivial, regardless of whether it is for standard algorithms or AI competitions. It requires constructing comprehensive test cases, writing reference solutions, and verifying that the intended approach is correct. I have contributed as a problem-setter for various algorithmic competitions, including Indonesia’s National Olympiad (2013–2015) and \href{https://icpc.global/ICPCID/XYTDHW4JI15A}{multiple ICPC events}. In fact, an \href{https://apac.icpc.global/assets/competition/problemset.pdf#page=23}{ICPC problem} I designed for the 2025 competition was inspired by the extremely long Gala Dinner queue at EMNLP 2024.

I was also appointed to the Scientific Committee of several prestigious competitions. Our responsibilities included ensuring that the problem sets were fair and challenging, testing solutions, writing post-contest editorials, and overseeing the competition in general. Specifically, I served on the scientific committees for several national-level contests in Indonesia in 2016 and 2017. I also served for \href{https://codeforces.com/blog/entry/59727}{TOKI-Open 2018}, a contest used to select Indonesia's IOI representatives which is also open to international participants. Furthermore, I served on the Host Scientific Committee for \href{https://ioi2022.id/data/ioi-2022-hsc-report.pdf#page=9}{IOI 2022}.

As AI becomes more prominent globally, it has become a recognized olympiad field (IOAI). I served as a problem setter and jury member for the 2025 IOAI in Beijing. Similarly, I co-organized Indonesia’s first pilot of the National AI Olympiad in 2025 together with some members of the informatics olympiad community. My role involved designing problems, evaluating solutions, and organizing both the online qualification and on-site finals. Looking ahead, I will continue this work as a member of the Host Scientific Committee for the IOAI 2026 in Abu Dhabi.

I am happy to be involved in this community and plan to continue doing so. I personally find that the rigor of competitive programming significantly enhanced my technical skills and helped me secure jobs before joining academia. While this might seem separate from my NLP career, I am happy to continue contributing in this area. Furthermore, the fields are actually becoming more connected now that we have the AI Olympiad. Additionally, recent AI and LLM research frequently uses IOI problems as a benchmark for reasoning capabilities.
\section*{Teaching Statement}



%I was fortunate enough to be selected as Indonesia’s delegate at the IOI and won the silver medal for the country during my high school days. Since then, I have been officially teaching and mentoring students in preparation for the informatics olympiad, whether at the national level or the international level. Therefore, I’m accustomed to teaching and mentoring students on advanced algorithms and data structures, especially aimed at informatics olympiads.

% At MBZUAI, I co-designed NLP801 from scratch. Since MBZUAI was new, this course too was among the courses that had to be designed for our first NLP cohort. This course is an overview of advanced NLP aimed at the PhD level that covers all the basics in deep learning with NLP and encompasses broad NLP topics, including multilinguality, efficiency, multimodality, retrieval, and more. Since the course is PhD-specific, we combine traditional teaching, paper reading and discussions, and hands-on projects. I also co-teach NLP702 the following year, Advanced NLP for Master’s Students. In this course, I cover efficient NLP as well as some aspects of multilingual NLP. In 2025, I became the main instructor for NLP702.
My approach to academic teaching is fundamentally rooted in my background in competitive programming, first as a participant and later as a coach. In 2007, I started my coding journey and I was fortunate to win a medal at the national level in 2008, followed by a \href{https://stats.ioinformatics.org/people/132}{silver medal in IOI 2010}, one of the most prestigious coding competitions internationally. In 2008, I led the coding club in my high school, often coaching my friends and juniors. This continued during my undergraduate studies, where I trained peers and juniors specifically for competitions.

Training for a competition is intensive and often requires long periods of learning. This experience taught me that engaging teaching is a must; maintaining students' motivation is key to better learning outcomes. In competitions, our goal is for the student not only to know high-level ideas but to deeply understand the algorithm so that they can actually implement it.

This paradigm carries over to my academic teaching. I try to deep-dive into topics rather than just giving a high-level overview. Since mathematical formulas can be intimidating, I prioritize detailed diagrams to build intuition first. However, I still deep dive into every specific component of that diagram. Whether it is the attention head in a Transformer, the rank decomposition in LoRA, or the loss function in Knowledge Distillation, I ensure students understand the mechanism rather than treating it as a black box. I want students to be able to employ the algorithms, not just discuss them. This approach is reflected in my course reviews, where students noted that I can explain complex concepts in "newbie terms" yet remaining technically rigorous, as reflected by students as being "technical".

I also ensure that the teaching is interactive and engaging. I strive to make the classroom less intimidating by incorporating humor and keeping the mood light. I design assessments that are fun and interactive. For example, in my recent graduate courses, I replaced standard final presentations with a poster session, mimicking a real academic conference where students actively discussed their work. For the upcoming undergraduate Algorithms course, I am designing a multiplayer game framework where students will implement algorithms to control agents that fight against one another. I designed this task to be accessible but scalable: implementing standard algorithms secures a good grade, while competitive mastery requires creative, complex strategies. This ensures the project challenges top students without overwhelming the rest.

Finally, I try to personalize the learning experience to engage more with the students. I often ask questions to benchmark understanding and adapt if topics have not yet been mastered. I check the student's status and provide extra sessions for those who need more help. For instance, in NLP801 (2023) and NLP702 (2024), I noticed that some students struggled with coding, particularly those coming from non-computer science backgrounds. To address this, I organized extra hands-on tutorials specifically on Python and PyTorch.

\subsection*{MBZUAI Teaching}

\paragraph{NLP801 - 2023} I joined MBZUAI right when the NLP department was formed and was appointed as the main instructor for NLP801, an introductory course for PhD students. Since the course was new, I had to design it from scratch, with the help of Thamar Solorio. As the program was new, we only had 6 students. I designed the course with the assumption that the PhD students were already familiar with NLP basics; hence, it was more advanced from the get-go and focused on discussions of current trends. However, along the way, I noted that some students had zero background in NLP, which required me to adjust the teaching on the go. This provided a learning opportunity to improve my upcoming teaching.

\paragraph{NLP702 - 2024} In the next term, I was tasked to co-instruct NLP702, a master-level course on advanced NLP covering modern topics such as transformers. However, I was not the lead instructor and taught only a minority of the sessions. Note that at MBZUAI, courses are often taught by multiple lecturers. Unfortunately, this course received mixed feedback. While students expressed concerns about the depth and repetitiveness of the material in the broader course, the feedback for my specific sessions was excellent. In fact, the course evaluations explicitly suggested that I should take on a larger portion of the teaching load.

\paragraph{NLP702 / NLP806 - 2025} In the next term, I was trusted to take the lead on NLP702. The class was combined with PhD students (NLP806), so the class size was even larger. I updated the materials to remove redundancy and to include more recent advancements, such as State-Space Models, and more practical advancements in distributed training of Large Language Models, interpretability, and more. I also taught the majority of the sessions. Feedback improved significantly from the previous year and the students were generally positive about the course. My personal teaching feedback also improved.

\subsubsection*{Teaching Feedback}

The following is my course feedback, and it shows that the students gradually increases. It also shows the improvement from NLP702 after I took the lead and improve the course. I also highlight positive feedback, including comments that students submitted in the negative feedback section.

\begin{table}[h]
\centering
\begin{tabular}{ l c c p{2.8cm} p{2.8cm} p{2.2cm} }
\toprule
Course & Term & Size & Course Rating \newline (out of 10) & Lecturer Rating \newline (out of 5) & University \newline Avg \\
\midrule
NLP801 (main lecturer) & Fall 2023 & 6 & 8.00 & 4.5 & N/A \\
\midrule
NLP702 (secondary lecturer) & Spring 2024 & 26 & 7.20 & 4.70 & N/A \\
\midrule
NLP702 (main lecturer) & \multirow{2}{*}{Spring 2025} & 35 & \multirow{2}{*}{8.46} & 4.55 & 4.39 \\
NLP806 (main lecturer) & & 16 & & 4.75 & 4.39 \\ 
\bottomrule
\end{tabular}
\caption{Course and instructor feedback score}
\end{table}



\begin{table}[h!]
    \centering
    \begin{tabular}{p{0.2\textwidth} p{0.75\textwidth}}
      
\hline
        Course & Feedback \\
          
\hline
        NLP801 (2023) & The instructors helped accelerate the learning curve for the taught students, insuring that all students catch up to the material that is being taught \\
          
\hline
        
        NLP702 (2024) & Dr. Alham's part especially during his session's labs because we learn everything from scratch and this is very helpful to enhance the knowledge practically and theoretically. \\
          
\hline
        NLP702 (2024) & Lectures of Dr. Alham! They were very useful, very clear, very interesting and relevant. For example, he was the first lecturer in my life who explained how GPU's work when we train the model, etc. \\
          
\hline
        NLP702 (2024) & Dr. Alham Fikri Aji's part. He got actual topics such as Adapters, Lora, etc. \\
          
\hline
        NLP702 (2024) & Suggest to Dr. Alham Fikri Aji to take bigger part of the course \\
          
\hline
        NLP702 (2025) & A lot of professors struggle to say "i don't know" when asked a question they don't have the answer to. Dr. Alham is one of very few professors that is comfortable and confident saying he doesn't know instead of answering something vague and it has been incredibly refreshing, especially given how technically strong he is. As someone with massive impostor syndrome, this normalization of not knowing everything is really important and helpful and I find myself less likely to try to cover my own tracks with dodgy answers when someone asks me a question I can't answer. \\
          
\hline
        NLP702 (2025) & Dr. Alham is the best! He can deliver complex concepts in a way that students understand. The pace of teaching is good. Funny. It would be better if he takes all parts of NLP702 next year :) \\
          
\hline
        NLP702 (2025) & he could explain everything in newbie terms, so i could understand easily \\
        
\hline        
        NLP702 (2025) & I feel like Professor Alham teaching is really really good, he is teaching about relevant stuff on NLP (AND IN DETAIL TO THE TECHNICAL DETAIL) \\
\hline
        
        
    \end{tabular}

    \caption{Some cherry-picked feedback.}
\end{table}

\subsection*{Advising and Mentoring}

My group organization is primarily flat. Students, postdocs, and assistants can easily come to my office or ping me via Slack or Discord to discuss any topic. Likewise, collaboration is encouraged; I urge them to discuss ideas with one another rather than working in isolation. Students frequently drop by my office to discuss research or simply to chat. This structure ensures I remain involved in all research projects and hands-on with many of them. I strive to avoid a purely managerial role where the research details are obscured, or letting the group become so hierarchical that I lose touch with my students' work.

To date, I have advised 5 PhD students, 9 MSc students, and several research assistants and visitors. I have supervised one postdoc, and another is joining soon. Within the MBZUAI NLP department, this represents a significant group size, particularly for an Assistant Professor.

Beyond MBZUAI, I actively collaborate with and advise external students, primarily BSc and MSc students from Universitas Indonesia and ITB, totaling 15 students so far. Several of these students have continued their studies at MBZUAI; I view this as a valuable investment in the talent pipeline that also allows me to contribute back to my country. While maintaining a flat structure, I encourage PhD candidates to mentor master's or undergraduate students to develop their own mentorship skills, though I remain actively involved in these projects as well.

My expectations for PhD students extend beyond the completion of a thesis. My goal is to mold them into exceptional researchers equipped to secure strong positions after graduation. Consequently, my mentorship goes beyond technical supervision. We frequently discuss career strategy and professional growth to ensure they are fully prepared for the field.

I do not demand a specific number of publications, though I do expect students to publish to avoid the "red flag" of a zero-publication track record. However, publication count is not the main focus. In an era where tens of thousands of papers are submitted each cycle, adding one more unrecognized paper is of limited value. Instead, I ask my students to aim for influential and impactful work, even if it is only one single paper.

While impact is harder to quantify, I define success through indicators such as community discussion, dataset or model adoption (downloads/usage), and citations by other influential works. I believe that this approach is far more valuable than accumulating dozens of unrecognized publications in a sea of papers.

I recognize the critical importance of internships. Securing an internship at a top tech company or reputable research lab is a mandatory requirement for the members of my group. I actively assist students by providing internal references to tech companies and connecting them with colleagues at various institutes. In addition, I support them with CV review and interview preparation when necessary.

\subsection*{Future Teaching and Mentoring Plan}

One of the challenges of teaching at a relatively new university is the limited number of students. Nevertheless, I see MBZUAI growing rapidly and having more students in the near future, so I am very happy to advise more PhD and MSc students from MBZUAI.

In terms of teaching, I will continue to teach NLP702/NLP806 at MBZUAI. MBZUAI has also grown as a university and now we have our first cohort of undergrads of around 100 students. The university trusts me to lead teaching this first cohort on Algorithms and Data Structures and Algorithms in 2026. 

Beyond that, I am proactively building an undergraduate team to compete in the ICPC, a highly prestigious international university-level competition in competitive programming. I recruited 15 top students and organized several selection contests. I now have 2 teams, each consisting of 3 students, and I am coaching them for the ICPC regional contest this coming December. Although ICPC training and DSA are not closely related to NLP, this work has been especially rewarding because it reconnects me with my long-standing passion for algorithms and coding.